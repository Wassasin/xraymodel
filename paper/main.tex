\documentclass[a4paper,10pt]{article}
\usepackage[british]{babel}
\usepackage[T1]{fontenc}
\usepackage{url}
\usepackage[a4paper]{geometry}

\usepackage{listings}
%\usepackage{tabularx}
%\usepackage{graphicx}
%\usepackage{rotating}
%\usepackage{float}
\usepackage{xspace}
%\usepackage{enumerate}
%\usepackage{diagbox}

% Math packages
%\usepackage[sc]{mathpazo}
%\usepackage{amsthm}
%\usepackage{algorithmic}
\usepackage{amsmath}

\lstset{basicstyle=\footnotesize}

% Document properties
\title{Modelling and Analysis of an Interventional X-Ray System with NuSMV2}
\author{
	Wouter Geraedts \\ \small{\texttt{w.geraedts@student.ru.nl}} \and
	Ko Stoffelen     \\ \small{\texttt{kostoffelen@student.ru.nl}}
}
\date{July 3, 2013}
\linespread{1.05}

% Terminology
\newcommand{\NuSMV}{NuSMV2\xspace}

% Math
\newcommand{\LTLG}{\mathbf{G~}}
\newcommand{\LTLF}{\mathbf{F~}}
\newcommand{\LTLX}{\mathbf{X~}}
\newcommand{\LTLU}{\mathbf{~U~}}
\newcommand{\LTLO}{\mathbf{O~}}
\newcommand{\LTLY}{\mathbf{Y~}}
\newcommand{\LTLH}{\mathbf{H~}}
\newcommand{\LTLS}{\mathbf{~S~}}
\newcommand{\disjall}[2]{\mathop{\bigvee}\limits_{#1}^{#2}}
\newcommand{\conjall}[2]{\mathop{\bigwedge}\limits_{#1}^{#2}}
\newcommand{\imply}{\rightarrow}
\newcommand{\doselow}{\textrm{dose\_low}}
\newcommand{\doseno}{\textrm{dose\_no}}
\newcommand{\dosehighseq}{\textrm{dose\_highseq}}
\newcommand{\dosehighone}{\textrm{dose\_highone}}
\newcommand{\lowsignal}{\textrm{low\_signal}}
\newcommand{\errorany}{\textrm{error\_any}}

\begin{document}
	\maketitle

	\section{Introduction}
	At Philips Medical Systems, an interventional X-ray system is being developed.
	Our goal is to make a NuSMV2 model based on a requirements sheet and some illustrating movies, that satisfies all these requirements.

	\section{Requirements}
	All atomic propositions correspond with variables or macro definitions in our model.
	
	\begin{enumerate}
		\item There should be no X-ray if there is no request for it.
			\begin{align*}
				(\LTLG !\errorany) \imply & (\LTLG \\
					& ((\doselow \land \LTLY \doseno) \imply \LTLY (\doseno \LTLS \lowsignal)) \land \\
					& ((\dosehighseq \land \LTLY \doseno) \imply \LTLY (\doseno \LTLS \textrm{i.signal = StartSeqHigh})) \land \\
					& ((\dosehighone \land \LTLY \doseno) \imply \LTLY (\doseno \LTLS \textrm{i.signal = StartOneHigh})))
			\end{align*}
		\item There should be no X-ray of a certain type on an Xplane if there is an error on that Xplane for that type.
		\item If there are no errors and no competing requests, then a request for X-ray should lead to the requested X-ray.
		\item The generated X-ray should conform to the current user request. For instance, if during a High dose, the request LowFrontOn, LowFrontOff occur then there should be no Low dose after the end of the High dose. Similarly, if while finishing a sequence of Low dose images, the requests StartOneHigh, StopOneHigh should not lead to High dose after the end of the Low dose.
		\item After a request for High (Sequence or One), no other High request can result in a High dose until all requests are released. In other words, only the first High request can result in High dose and only when that request has been finished (and there are no other pending High requests) a new request can result in a High dose.
		\item Once a Low dose (on any Xplane) has been requested, no other Low request can result in a Low dose before all requests are released. In other words, only the Low request can result in a Low dose and only when that request has been finished (and there are no other pending Low requests) a new request can result in a Low dose.
		\item Requests for High dose have priority above Low dose requests.
		\item A High request (PrepSeqHigh or StartOneHigh) without an error interrupts any Low dose. After this end of such a High dose interrupt, an interrupted Low request or a new pending Low request may continue.
		\item A High dose (prepare or start) is never interrupted by a Low dose request.
		\item The system is not prepared or a dose is started if there is an error for the corresponding type and Xplane.
		\item For Low dose a degraded mode is used in case of an error, that is, if Xplane Both is requested then
			\begin{enumerate}
				\item only Xplane Front is used if SideLowError holds;
				\item only Xplane Side is used if FrontLowError holds.
			
			\end{enumerate}
		\item There is no degraded mode for High dose; if there is an error on one the Xplanes no High dose should be generated.
		\item For a request it is checked only once whether an error exists; if there an error, the request is discarded. The exception is an interrupted Low dose; then the error condition is checked again when the Low dose may resume.
		\item
			When an error occurs during the generation of X-ray for the corresponding type (independent of the Xplane), then any X-ray of that type is stopped immediately.
			A new generation of this type of dose may only occur when all requests for that type have been released and there is no error.
			For instance, High dose X-ray on Front is stopped when SideHighError occurs.
			High dose may only be generated when all High requests are released an all High errors have been resolved.
		\item It should not be possible to select an erroneous Xplane for High dose.
			In the actual system the user would receive a notice, that the selected Xplane does not work.
			Because we do not have to implement this notice (mail with Jozef from 2013-06-23), this requirement does not say anything about our model.
			Thus we did not implement this LTLSPEC.
	\end{enumerate}
	
	Most of these LTL specifications take a very long time to compute in NuSMV2.
	This shows that with our model we reach the boundaries of what is feasibly computable with NuSMV2.
	However, during testing we noticed that the \textsc{-dynamic}-flag of NuSMV2 works exceptionally well to reduce computing time and required memory.
	With this flag, NuSMV2 adapts the variable ordening to suit each seperate LTLSPEC.
	
	\section{Our model}
	We modelled the pedals and the handswitch of the interventional X-ray system as a stream of input signals.
	An input module can fire these signals at random.
	
	During the first `stage', the model checks if the signal is valid.
	Some signals can directly be thrown away, like a \texttt{LowFrontOff} signal when there has never been a \texttt{LowFrontOn} signal.
	If it is valid, then the signal is converted to a machine instruction.
	Contrary to UPPAAL, NuSMV2 does not support atomic assignments for multiple variables.
	We've implemented this separate machine instruction because it is not possible to consume a signal and process it in the same transition.
	
	This is especially required in case of the \texttt{SelectXPlane}-signal.
	This signal is supposed to change the internal \texttt{next\_direction}-variable \emph{once}.
	There is no way to change this direction, and then mark the signal as \emph{done} in the same transition.
	Thus, we first translate this signal to specific \texttt{SelectPlane}-instructions, before marking the signal as \texttt{resolved}.

	Third, internal variables like \texttt{next\_direction} and \texttt{high\_sequential\_mode} are set to the values of what the machine \emph{wants} to do next.
	Fourth, variables of the output module are set to their correct values.
	Finally, the instruction is marked \texttt{done} and we allow for new signals to be interpreted.
	This whole cycle is necessary, unfortunately, because NuSMV2 does not support atomic operations.
	
	A seperate error module can raise random errors during any transition.
	Because the machine has to remember if there has been an error after a pedal has been pressed, it too has four `blocked' variables.
	These are only set to false after the error has gone and when the relevant pedals are no longer pressed.
	The addition of errors also made us add seperate directions for the degraded mode: \texttt{DegradedSide} and \texttt{DegradedLow}.
	Some additional comments on specific lines can be found in our model.

	\section{Reflection}
	We had some troubles while making this assignment.
	The most difficult part was dealing with the requirements sheet.
	The requirements are not exhaustive, some are ambiguious and some are simply vague.
	This is partly due to that the system itself is very complicated and in some situations counter-intuitive.
	The videos help a bit, but also do not cover all possible situations.
	It would have been helpful to have some sort of black box to test situations on.
	If this would have been a real world scenario, there would naturally not be a black box available.
	
	The modeling tool itself was useful, because NuSMV2 quickly finds situations that one has not thought of when designing the model / system.
	However, the assignment is very similar to the previous assignment.
	So we did not learn anything new about temporal logic or model checking.
	Something that we had some issues with, was that NuSMV2 cannot atomically assign multiple variables in one transition.
	All the internal administration of the model has to be done in a number of steps, where errors can pop up in any transition in any combination.
	This hugely increases the state space and makes it very hard to check some properties.
	
	In retrospect using UPPAAL would've been a better choice for this model, because UPPAAL does support atomic assignments.
	With UPPAAL however, we would not have been able to define the same \texttt{LTLSPECs} for the model due to less advanced LTL support in UPPAAL.
	This could be worked around by implementing the appropriate Observer automatons.
\end{document}
